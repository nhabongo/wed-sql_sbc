\documentclass[12pt]{article}

\usepackage{sbc-template}

\usepackage{graphicx,url}
\usepackage{algorithmic}
\usepackage{algorithm}
%\usepackage[brazil]{babel}   
%\usepackage[latin1]{inputenc}  
\usepackage[utf8]{inputenc}
\usepackage[T1]{fontenc}
\usepackage{fancyvrb}

\renewcommand{\algorithmicrequire}{\textbf{Input:}}
\renewcommand{\algorithmicensure}{\textbf{Output:}}
\providecommand{\keywords}[1]{\textbf{\textit{Key words: }} #1}
     
\sloppy

\title{WED-SQL: A SQL Extension for Implementation of Business Processes}

\author{Bruno Padilha\inst{1}, André Luis Schwerz\inst{2}, Rafael Liberato Roberto\inst{2},\\ Calton Pu\inst{3} and João Eduardo Ferreira\inst{1}}

\address{IME -- University of São Paulo (USP) \\São Paulo, SP, Brazil
\nextinstitute
Federal University of Technology - Parana (UTFPR) \\Campo Mourão, PR, Brazil
\nextinstitute
CERCS, Georgia Institute of Technology \\ Atlanta, GA, USA
  \email{\{bruno,jef\}@ime.usp.br,\{andreluis,liberato\}@utfpr.edu.br, calton@cc.gatech.edu}
}

\begin{document} 

\maketitle

\begin{abstract}
Despite the significant evolution of the design and implementation of business process models, a transactional approach that evolves an incremental and adaptive strategy remains an important challenge to be overcome for database management systems. The complexity of implementation based on traditional frameworks - such as BPEL, Process Algebra, and Petri Net - is too high, especially to deal with semantic exceptions. In this paper, we present the WED-SQL, a distributed framework that provides a reliable and efficient way to implement business processes by integrating WED-flow concepts into the PostgresSQL RDBMS. It is a transactional framework that benefits from the SQL language to specify the WED-flow processes. 
\end{abstract}
\keywords{WED-flow, Business Process Management, Long Lived Transactions, PostgreSQL, Client-Server Architecture, Distributed Systems}
     
\section{Introduction}

Most  modern business process management software systems are prone  to recurrent structural modifications during its life cycle. These  changes are needed to incorporate new requisites to the business processes, once it is often impossible  to come up with all requirements upfront in the modeling phase of project - some of them only appear after the software is running in production. Furthermore, it is too expensive and time consuming to try to predict all the possibles paths  and its outcomes, which may arise as exceptions that must be dealt with. Over time, these modifications usually lead to code deterioration - compromising the system overall performance - and exponential increase of maintenance costs.

Classic business process specification models, for example WS-BPEL, Process Algebra, and Petri Net, focus primarily on behavioral interactions and relations between processes, pushing data analysis into the background and overlooking its relevance for the project. Therefore, none of these models exactly fits the needs of design dynamic business process management systems, which may result in costly system adaptations and maintenance.

The WED-flow approach~\cite{ICWS12}, in contrast to the classical models, captures both the inter-process relations  and the data generated by their interactions with the same relevance. With this in mind, the WED-flow allows systems architects to think in terms of data states and transitions, similarly to a finite automata designing. Modifying or adding new business rules to the project, which can often be translated to creation of new processes, is a simple matter of defining new data states and conditions for their transitions. This captures the true nature of dynamic processes and simplifies the incremental evolution  of the whole project. Exceptions are also easier to be handled and integrated to the project, once this approach is based on transactional properties and provides several transactional recovery mechanisms.

 In order to provide a standard implementation as well as simplifying its software development, in this paper  we present  the WED-SQL, a WED-flow relational framework for implementation of business process. The main purpose of WED-SQL is to implement all WED-flow definitions~\cite{ICWS12} and encapsulate the control structure, what includes the transactional control, exceptions handling, data states detection, state transitioning,  and ensure data integrity. This framework effectively is an abstraction layer that allows developers to put more effort
on business logic, which results in better WED-flow projects and software that makes better use of reusable code. 

 Once the WED-SQL framework has been developed  inside the PostgreSQL~\cite{PSQL} Relational Database Management System (PostgreSQL RDBMS), it is highly fault tolerant, has full transactional support while WED-flow properties and definitions are specified using the SQL language. By employing well consolidate  technologies - widespread in the computer science area - this framework targets to disseminate the adoption of the WED-flow approach on business process management and software design. It is a reliable and scalable tool that is able to provide the flexibility demanded by modern transaction control systems. 

 The remaining of this paper is structured as follows: Section ~\ref{sec:arch} presents WED-SQL architecture and its internal data structures. Section ~\ref{sec:guts} briefly describes how the system works. Following, a detailed description of transactional control and the communication protocol used for external communication is given in Section~\ref{sec:prot}. Finally, in Section~\ref{sec:fut} we present our conclusions and how we plan to improve and expand this framework.  

\section{WED-SQL: general view and architecture} 
\label{sec:arch}


To better understand what the WED-SQL does, first one needs to understand the basic definitions of the WED-flow paradigm. 
A WED-flow application is composed of a set of WED-attributes $\mathcal{A} = \{a_1,a_2,a_3,\ldots,a_n\}$, a set of WED-states
$\mathcal{S} = \{s_1,s_2,s_3,\ldots,s_m\}$ where $ \forall s_i \in \mathcal{S}, s_i = (v_1,v_2,v_3,\ldots,v_n)$
is a n-tuple where each $i\in[1,n], v_i$ is a value for $a_i \in \mathcal{A}$ and the size of $s_i = |\mathcal{A}|$, a set of
WED-conditions $\mathcal{C}$, a set of WED-transitions $\mathcal{T}$ and a set of WED-triggers $\mathcal{G}$. A WED-condition
$c \in \mathcal{C}$ is a predicate over $\mathcal{S}$, that is, we say that $s \in \mathcal{S}$ satisfies $c$ if the values
of $s$ makes $c$ true. A WED-transition is a function $t: \mathcal{S}\rightarrow\mathcal{S} \in \mathcal{T}$ that receives 
as input a WED-state $s$ and returns another WED-state $s'$. Finally, a WED-trigger $g = (c,t)$ where $g \in \mathcal{G},
c \in \mathcal{C}, t \in \mathcal{T}$ is a 2-tuple that associates a WED-condition $c$ with a WED-transition $t$. Therefore,
an instance of a WED-flow application starts with an initial WED-state $s$ that should match some WED-condition $c$ to fire
the WED-trigger $g$ that, in turn, will initiate a WED-transition $t$, which must complete by some predefined timeout,thus 
generating the next WED-state. This keeps going until a final WED-state is reached, when this instance is then finalized. 
Roughly speaking, a WED-flow application is a transactional data state transitioning machine.

The WED-SQL has been designed to manage business process in distributed environments (e.g., web-services composition) and support execution of long-lived transactions. Its architecture is based on the client-server model and its two main components are described as follows. The \emph{WED-server} is responsible for both transactional and flow controls. This includes transitions coordination, time-out management for transactions to complete, matching between WED-states and WED-conditions, and WED-triggers activation.  On the other hand, each \emph{WED-workers} is responsible for the execution of a
specific WED-transition. 

\subsection{WED-worker and WED-server}
Fundamentally, the WED-server is a PostgreSQL extension package that includes  Database Triggers, control tables and Stored Procedures~\cite{NAV}. We opted for PostgreSQL as the WED-SQL foundation mainly due to its following aspects: open source project, catalog-driven operation, and support for user-written code using low and high level programming languages (e.g., C, Python, Perl, TCL and SQL itself).

\par Standard RDBMS need to store some metadata about internal structures of tables, columns, indexes, etc. This data is kept in what is know as system catalogs. PostgreSQL not only stores much of its control data in these catalogs but also bases its operations on then, in what is called a catalog-driven operation. Since they are readily  available to users as ordinary tables, it is fairly easy to modify the PostgreSQL behavior. This property is specially useful to allow the WED-server to manage timed transactions related to WED-transitions and exception  handling. 

\par As previously mentioned, users can write PostgreSQL modules using a few different languages. The most simple and straightforward of them is a SQL dialect called pgSQL, which is basically SQL with variables and flow control structures. These modules can also be written in C language and loaded as dynamic libraries, providing the developer with ultimate logic control and higher performance. Despite  these two language having native support in PostgreSQL, sometimes pgSQL is not expressive enough for complex logic,
while C is too low level and requires an overload of technical details for implementation of  most tasks. The PostgreSQL documentation recommends using a higher level language other than C unless it is explicitly needed. Fortunately, the PostgreSQL also offers support for Python, a much more versatile and expressive programming language than the two former. Due to  the dynamic nature of WED-flow models, Python has been proven the perfect tool to do most of the job on the WED-server side, only recurring to C when critical or where an extra performance is absolutely necessary. And, of course, SQL statements are everywhere in both WED-server and WED-workers. 

\par The role of the WED-workers is to perform the WED-transitions. They must connect to the WED-server and "ask" for a pending job before opening a new transaction. All WED-transitions are encapsulated in a single transaction with a time limit to  complete. Otherwise, they are automatically aborted by the WED-server. Each WED-transition must have at least one WED-worker associated and, depending on the workload on the server side, can have many more. On the other hand, each WED-worker is  specialized in execute a specific WED-transition. The maximum number of WED-workers simultaneously transacting depends  on the number of concurrent connections that the WED-server is able to keep alive.

\subsection{Data Model}

\begin{figure}[!t]
\centering
\includegraphics[width=2.5in]{ER.png}
\caption{Approximate WED-server's Entity-Relationship model}
\label{fig_er}

\end{figure}
 The diagram depicted in Figure \ref{fig_er} represents the data model used by the WED-server. It is not a true Entity-Relationship model once the relation between the \emph{WED\_attr} and \emph{WED\_flow} (represented by the arrow in the diagram) is 
managed by a stored procedure rather than by the RDBMS. This is necessary to comply with the dynamic nature of the WED-flow approach, which allows the system designer to evolve the system on-the-fly, for instance, including new WED-attributes and 
WED-conditions. In this case, the system's data structure may need to be modified accordingly. 

\par WED-attributes are represented by the entity \emph{WED\_attr} using the following attributes:
\begin{itemize}
\item $aname$: Name of the WED-attribute and also the primary key;
\item $adv$: Default value for this WED-attribute.
\end{itemize}

The \emph{WED\_flow} entity represents the instances of a WED-flow application. Besides the identifying attribute (\emph{wid}), it has a extra one for each WED-attribute defined for a given application, in other words, for each row inserted in \emph{WED\_attr}
a column representing the new WED-attribute  $aname$  is added to \emph{WED\_flow}.

\par The weak entities \emph{Job\_Pool} and \emph{WED\_trace} results from the relation between the entities \emph{WED\_flow} and  \emph{WED\_trig}. Identified by the ternary relationship \emph{Match}, \emph{Job\_Pool} represents pending WED-transitions, i.e., the WED-transitions that are  awaiting to be processed by a WED-worker, using the following attributes:

\begin{itemize}
\item $wid$: Foreign key that identifies a WED-flow instance;
\item $tgid$: Foreign key that identifies which WED-trigger fired this WED-transition;
\item $trname$: Name of the WED-transition to be performed;
\item $lckid$: Optional parameter used by WED-workers to be able to identify themselves. This attribute will be used for authentication purposes in the future work;  
\item $timeout$: Reference to the attribute of same name in entity \emph{WED\_trig}. It is the transaction time limit defined for this WED-transition.  
\item $payload$: It represents the WED-state that satisfied the WED-condition associated with this WED-transition. 
\end{itemize}

Identified by the ternary relationship \emph{Register}, \emph{WED\_trace} represents the execution history for all WED-flow instances in a WED-server. Its attributes are:

\begin{itemize}  
\item $wid$: Foreign key that identifies a WED-flow instance;
\item $state$: WED-state that fired the WED-transitions represented by the multivalued attribute \emph{trf};  
\item $trf$: WED-transitions fired by the WED-state in \emph{state} attribute;
\item $trw$: WED-transition that committed the respective WED-state in \emph{state}. A null value indicates a initial WED-state for the given WED-flow instance;
\item $status$: Labels \emph{state} accordingly to its status: "F" indicates  a final WED-state, "E" indicates that an exception has occurred and "R" means a regular WED-state;
\item $tstmp$: The exact moment when this record was created. It can be used to retrieve the execution history of a WED-flow instance in chronological order. 
\end{itemize}

Finally, the \emph{WED\_trig} entity represents the WED-triggers, i. e. , the associations of WED-conditions with WED-transitions, using the following attributes:

\begin{itemize}
\item $tgid$: Primary key; 
\item $tgname$: Optional parameter that indicates the name of the WED-trigger; 
\item $enabled$: Allows a WED-trigger to be disabled;
\item $trname$: Unique name of the associated WED-transition;
\item $cname$: Optional parameter that may  be used to name the associated WED-condition; 
\item $cpred$: WED-condition's predicate . Accepts the same syntax and tokens allowed in a SQL's \emph{WHERE} clause;
\item $cfinal$: Marks a WED-condition as final. Although only one is allowed, multiple stop conditions can be defined by combining them with the logical operator \emph{OR}; 
\item $timeout$: Time limit to perform the WED-transition. 
\end{itemize}

This conceptual data model is mapped to five database tables inside the WED-sever, one for each entity previously explained, using the same name. However, the WED\_flow table is special. It will be dynamically modified every time a new record is added, modified or removed from the WED\_attr table. For example, after inserting a new record into WED\_attr, a new column identified by  the same name of this new WED-attribute, will be added to WED\_flow. Each row in WED\_flow represents a distinct application instance. 

\section{How it works}
\label{sec:guts}
\begin{figure}[!t]
\begin{Verbatim}[fontsize=\tiny]
BEGIN;

INSERT INTO wed_attr (aname, adv) VALUES ('a1','ready');
INSERT INTO wed_attr (aname) VALUES ('a2'),('a3');

INSERT INTO wed_trig (tgname,trname,cname,cpred,timeout) 
  VALUES ('t1','tr_a2','c1', $$a1='ready' and (a2 is null)$$, '3d18h');
INSERT INTO wed_trig (tgname,trname,cname,cpred,timeout) 
  VALUES ('t2','tr_a3','c2', $$a1='ready' and (a3 is null)$$, '00:00:30');
INSERT INTO wed_trig (tgname,trname,cname,cpred,timeout) 
  VALUES ('tf','tr_final','cf', $$a1='ready' 
          and (a2 is not null) 
          and (a3 is not null)$$, '00:00:10');
INSERT INTO wed_trig (cpred,cfinal) VALUES ($$a1 <> 'ready'$$, True);

COMMIT;
\end{Verbatim}
\caption{Defining a new WED-flow application}
\label{code_new}
\end{figure}
 
A new WED-flow application is created by defining a set of WED-attributes and a set of WED-triggers that associates a  WED-condition to a WED-transition. For the time being, this is accomplished,  by writing these definition in SQL as illustrated in Fig.~\ref{code_new}. In ongoing work, we are proposing a SQL language extension to express the WED-flow basic elements. 

\par Note that the expression of a WED-condition predicate (\emph{cpred}) uses the same syntax of the SQL clause \emph{WHERE}. In fact, this expression is employed \emph{as-is} for the WED-server while matching it against a WED-state. When surrounded by double \$ marks, special symbols like quotes don't need to be escaped in the predicate.

\par  Also note that, although the application's final condition statement is declared in the same WED\_trig table, it does not have a WED-transition associated. To declare a WED-condition as final, the designer  only needs to specify its predicate and set the value of the \emph{cfinal} attribute to true. If the final condition is missing, all instances of a given application will end up in an exception state. 

\par We encourage developers to encapsulate all definitions for a new WED-flow application into a single transaction. By doing so one can prevent a partially defined application in case of syntax errors, which may force the developer to manually truncate the system tables before running the script again. 

\par Although the WED-SQL enables multiple WED-flow applications to be defined in a single database, they can also be isolated from each other. Therefore, applications can be separated by some semantic meaning or even to restrict access to some specific  WED-attributes that must not be shared.

\subsection{Creating WED-workers}

Since WED-workers are client applications of the WED-server, they can be written in any programming language that is able to connect to the PostgreSQL database by simply implementing the WED-SQL's communication protocol (described in Section~\ref{sec:prot}). The WED-SQL provides one WED-worker implementation through a Python package named \emph{BaseWorker}. Figure~\ref{fig_ww} depicts an example of a WED-worker implemented with this package.

\begin{figure}[!t]
\begin{Verbatim}[fontsize=\tiny]
from BaseWorker import BaseClass
import sys

class MyWorker(BaseClass):
    
    #  trname and dbs variables are static in order to conform 
    #with the definition of wed_trans()
        
    trname = 'tr_aaa'
    dbs = 'user=aaa dbname=aaa application_name=ww-tr_aaa'
    wakeup_interval = 5
    
    def __init__(self):
        super().__init__(MyWorker.trname,MyWorker.dbs,MyWorker.wakeup_interval)
    
    # Compute the WED-transition and return a string as the new WED-state, 
    #using the SQL SET clause syntax. Return None to abort transaction
    def wed_trans(self,payload):
        print (payload)
        
        return "a2='done', a3='ready', a4=(a4::integer+1)::text"
        #return None
        
w = MyWorker()

try:
    w.run()
except KeyboardInterrupt:
    print()
    sys.exit(0)
\end{Verbatim}
\caption{WED-worker implementation in Python}
\label{fig_ww}
\end{figure}
The first step to write a WED-worker using the BaseWorker package is to import the module \emph{BaseClass}, which in fact is an abstract class that implements the communication protocol with the WED-server as well as manages all connections and transactions involved.  

\par  Next, a concrete class must implement the abstract method \emph{wed\_trans()} from BaseClass and initialize the following class attributes: 
\begin{itemize}
\item \emph{trname}: WED-transition name that will be carried out by this WED-worker. It must match the name used on the WED-flow definition;
\item \emph{dbs}: WED-server connection parameters complying with the \emph{psycopg2} driver format. At least the username, database name, and the WED-worker name are required;
\item \emph{wakeup\_interval}: Time interval in which the WED-worker execution is suspended (sleeping) if there is no new WED-transitions to be performed (explained in Section~\ref{not}); 
\end{itemize}

The wed\_trans() method works on a specific WED-flow instance at a time. It receives, as a parameter, the WED-state that  fired this WED-transition and returns a string that represents a new WED-state. This return value must comply with the same syntax used to specify a \emph{SET} clause of a \emph{UPDATE} SQL statement. An empty value can be returned to abort the transaction.

\par Finally, the concrete class can be instantiated and executed by invoking its \emph{run()} method.



\section{Transactional Management}
\label{sec:prot}

Each instance of a WED-flow application is  treated as a Long-Lived Transaction (LLT). Thereby WED-transitions may be regarded as \emph{SAGA} steps ~\cite{SGD87}. Therefore, the WED-server is responsible for ensuring the integrity of data states by enforcing the transactional consistency and that each instance will always end in a final WED-state. , When an exception is raised, the WED-server  must provide support for an external recovery mechanism, eventually leading this instance to a final WED-state. 

In the following sections, we discuss in details how the WED-server manages the timed transactions. We also present how to tackle some concurrency control challenges as well as explain the WED-SQL communication protocol. 

%DETALHES DE FUNCIONAMENTO (seriação,locks(),keep connection alive(pessimistic locking),excessoes, tras simultaneas,continuous query,protocolo de comunicaocao, balanço: coneccoes x demanda,payload pode nao ser o estado atual mvcc)
\subsection{Notifications}
\label{not}
%send notify vs continuous query

Event-driven computer systems can be described as a system that reacts in the face of a new event. In order to  accomplish that, event detection mechanisms are often integrated to such systems. Since a data event based system usually keeps its data stored in a database, data event detection is generally made by a polling operation, that is, by continually checking some data for changes. This database polling technique, known as \emph{Continual Query}~\cite{CQ}, mainly consists of  continuously run a hand tailored query for each targeted data event.     

When employing the Continual Query model in a system that is response time sensitive, one needs to keep in mind the computational cost of this approach, which in turn depends both on the number of distinct events to be detected and on how 
often they pop up. In order to implement this model efficiently, it is critical to have a good time estimation of events occurrence, especially in the presence of multiple distinct events, once there is an overhead cost involved in performing a query that cannot be disregarded. In the other hand, if this time estimation is unpredictable for some or all of the events, the suitable query must be executed as frequent as possible, otherwise threatening the system's real time response capability. Anyway, the Continual Query model is not well suitable to every data event driven system, which ultimately may lead to waste of computational resources, increasing the overall power consumption, and possible overloading the system.     
  
Since the WED-SQL is also a real time data event detection system - this functionality is indeed part of the WED-server main algorithm - it possible to take advantage of this and inform a specific WED-worker of the occurrence of its event of interest.     

The PostgreSQL provides a mechanism named  \emph{Asynchronous Notification}~\cite{AN} that can be used as an alternative to the Continual Query model. Via the command \emph{NOTIFY}, the RDBMS can asynchronously notify a client that  is listening to a communication channel. A client, in turn, can register itself via command \emph{LISTEN} on a specific  channel to receive notifications. Furthermore, the server can also send the client a payload message attached to each notification. 

The biggest advantage of the Asynchronous Notification model over Continual Query is that messages can be delivered instantly after the detection of an event since a client is  listening to a given communication channel. Moreover, since the client is not required to poll the server, it can be suspended until arriving a new notification, thus freeing computational resources as well as saving power.In the WED-SQL framework, these two mechanisms are employed in a complementary combination. 

 Under normal execution, WED-workers are notified whenever a new WED-trigger is fired. In other words, this happens  every time that a new record is inserted into the Job\_Pool table. In addition, the payload message sent together with the notification is exactly the new "job" generated, so the WED-workers do not need to go to the WED-server to fetch a task, saving time and the cost of an extra query. Every WED-transition, in turn, has a proper notification channel identified by its own name as recorded on the WED\_trig table. WED-workers subscribe themselves to his WED-transition channel and  go into suspended mode until they are awakened with a new notification. In case a notification is sent and there is nobody listening to that channel, it will be discarded.  

Once the WED-server and the WED-workers could be running on physically different machines, communicating over a network,  messages can get lost. In order to ensure that all WED-transitions are eventually performed as well as to minimize their waiting to run time,
that is, the time that a WED-transition sits into the Job\_Pool table waiting to be performed, WED-workers don't rely solely on asynchronous notifications. In fact, before they even register on a notification channel, their first step is to scan the Job\_Pool table for pending WED-transitions, and only then eventually goes into suspension state. With this in mind,
WED-workers can be also configured to wake up after some time suspended and go check for available tasks on the WED-server.

As previously mentioned, the message sent out to WED-workers carries a WED-state that is the same one recorded on Job\_Pool. By no means it should be regarded as the current state of a WED-flow instance. Remember that WED-transitions can be executed
in parallel and therefore modifications to an instance may happen even before a pre-fired WED-transition starts. This WED-state reflects an instance snapshot by the time when the WED-transition was fired. So, if the firing conditions are already known by  the WED-worker, what is the reason to send this data to then ? The answer is simple: efficiency. When a predicate of a  WED-condition is a disjunction of predicates, the WED-worker may need to know which ones were satisfied by the WED-state. Having this data before hand eliminates the need to query the WED\_trace table.

It is very important to note that a WED-worker should never rely on WED-attributes values that were not specified in the WED-condition's predicate, once, apart from being a project error, it may end up capturing spurious conditions or refusing to perform a legitimate one.

\subsection{Performing WED-transitions}
%wed-trans, why lock ?, types of locks , why advisory(pg_row_locks)
%locks are needed for: 1-keep track of wed-trans; 2-concurrency control between wed-workers (pg_locks, select for update nowait)

WED-workers can only perform WED-transitions properly recorded on the Job\_Pool table. Moreover, they must request the WED-server an exclusive lock on the task to be performed, and only proceed after this lock is granted. The WED-server rely
on this lock protocol to enforce each transaction to complete or abort within the time limits specified for each WED-transition. It also used to solve conflicts between concurrent WED-workers. 

One characteristic of a well designed WED-flow project is the presence of reasonable execution time limits for its WED-transitions. Although it is up to the project designers to define these time limits, it is the role of the WED-server to enforce these restrictions.
In order to accomplish that, it must first identify a WED-transition that is ready  to begin, and then keep track of its running time. It is worth to remember that each WED-transition is wrapped in a single transaction.

One way to force transactions to identify themselves is to require a exclusive lock when performing WED-transitions, effectively informing the WED-server of what task in Job\_Pool is going to be worked on. Another possible solution would be semantically identify the WED-transition based on the new WED-state that is about to be committed, but not without a 
few caveats. First, it could allow a transaction to stay alive much longer than necessary in the case of WED-transition that would be aborted anyway due to the timeout. Second, the WED-server would need to know which new WED-states are valid for each WED-transition to set, potentially harming its exception handling capability. Moreover, the locking solution can also be used to solve concurrency conflicts.    

The PostgreSQL provides client applications with several explicit locking mechanisms, just in case a more refined control over the data is required. These locks can be used in either table or row level and are automatically acquired for the RDBMS native operations. In the WED-SQL context, only row level locks are explicitly employed.  

The most common exclusive access row level lock is acquired using the SQL expression \emph{FOR UPDATE} by the end of a \emph{SELECT} command. A transaction that tries to lock a row that was previously locked with that lock,  will be either blocked and must wait until the end of the current transaction, or aborted if the expression \emph{NOWAIT} was presented after \emph{FOR UPDATE}. However, there is another type of lock named \emph{Advisory Lock} that can be used to achieve this same behavior..

Advisory Locks work on a semantic level, that is, they are only useful when they have some semantic meaning in the  application. In other words, the RDBMS does not enforce their use and nothing is really locked. It is up to the application to use them rightly. As an example in the WED-server, pending WED-transitions in the Job\_Pool table are uniquely identified by the attributes pair (\emph{wid},\emph{tgid}). WED-workers request Advisory Locks based on this composite key, which is granted only once for each pair until the transaction is over. When a lock is not granted for a WED-work, it knows that some other "sibling" is currently working on that WED-transition, even though the record is not really blocked on the database, and then it is free to look for another task. Once again, the WED-server will reject transactions that are not holding a valid lock. 
 
Given that the \emph{FOR UPDATE} approach seems to be the more straightforward way to manage WED-transitions, why did we decide to stick with Advisory Locks ? Well, one reason is purely technical and is based on how PostgreSQL works with different types of locks. Another reason is due to how the WED-state transitions is carried out by our framework. 

Regarding the technical aspects of the RDBMS, information about \emph{FOR UPDATE} locks are kept on disk instead of the RAM memory. Because of that, this data does not show up in system catalogs, and can only be obtained in real time using a third party module (pgrowlocks) that, besides being supplied by PostgreSQL itself, is rather inefficient. On the
other hand, information about Advisory Locks can be easily and efficiently, obtained directly from system catalogs.  

In the WED-SQL, a WED-transition begins on the Job\_Pool table and ends on the WED\_flow table. On this matter, the application is fully aware of the meaning of this lock protocol. Besides that, the FOR UPDATE lock is meant 
to be used to lock a record that is going to be updated and not to signal a change to be performed in another table. This could potentially compromise
data integrity of WED-states.Therefore, the solution based on semantic locks is simpler to be implemented, safer to be used, and a more elegant strategy.

\subsection{Transaction isolation and parallelism }

The PostgreSQL employs a multiversion concurrency control model (MVCC) to maintain data integrity. This means that each transaction sees a snapshot of the database taken right before it begins to run, which keeps them isolated from each other since no uncommitted 
data is shared between concurrent transactions. Another property of this model is that data read locks never conflict with data write locks, minimizing lock contention.

The ISO/ANSI SQL standard defines four levels for transaction isolation. In increasing order of restrictiveness are they:  \emph{Read Uncommitted, Read Committed, Repeatable Read e Serializable}

Transaction running in Read Committed level - default for all WED-server’s transactions - will view data committed by concurrent transaction, in the sense that two consecutive runs of a select statement may return a different set as a result, even though it is performed inside a  single transaction. This phenomena is know as \emph{nonrepeatable read}.
 
 Due to WED-state detection mechanism behavior and the MVCC model, concurrent WED-transitions do not need to be fully isolated  from each other. In addition, the order of their execution is never critical for a correctly designed WED-flow project.  For instance, if a WED-transition aborts after setting up a new WED-state, the possible fired new WED-transitions will never be seen by the WED-workers and the database will return to the previous consistent state.  

 Concerning the WED-workers, they are able to work in parallel on a same WED-flow instance until the final moment, when they need to update the WED-state. At this point, each transaction will be granted a lock to update this instance at a time, while the others must either wait or abort. During a normal operation of the WED-server, this last step of performing a WED-transition should be fast, therefore worthwhile for concurrent transactions to wait at least for some small amount of time before they eventually decide to abort. This waiting time to commit is bounded by the number of concurrent transactions waiting for  the lock multiplied by how long the WED-server takes to perform the update. Distinct WED-flow instances can be worked fully in parallel (See Figure ~\ref{fig_wf}).
   
\begin{figure}[!t]
\centering
\includegraphics[width=2.5in]{wed-flow2.png}
\caption{Two WED-workers processing the same instance may need to synchronize their update at the WED\_flow table}
\label{fig_wf}
\end{figure}

%escalabilidade (horizontal, vertical, consistencia transacional, pg_shard) 

%pgBouncer = breaks notify
%wasting connections
%pubsub (rabbitMQ, real event bus)
%pg process size in memory (~9MB)
%file descriptor per process
%(linguagem, gerenciador de conecções, ...)


\section{Future Work \& Conclusion}
\label{sec:fut}

The WED-flow approach has been evolving since its first proposal back in 2010. Previous and ongoing studies have raised new demands for a framework towards supporting  more advanced features presented in modern WED-flow models. Such features are no longer supported by the prototype available until now known as WED-tool~\cite{WT}. 

The WED-SQL framework has meticulously been developed  to supply these demands. Its integration to a RDBMS ensures that an application's flow  control is always performed under transactional rules and supported  by ACID properties. By adopting a client-server architecture, the WED-SQL   has been designed to run in a distributed environment and support thousands of parallel transactions. In summary, WED-SQL is not only the most advanced WED-flow framework but it also aims to spread the adoption of the WED-flow approach. It is a reliable and simple to use tool to both design and implement business process.

The next phases of this work will be focused on two main tasks: (i) create a declarative language that is able to natively express all elements of the WED-flow approach; and (ii) develop a connection manager module to reduce the computational resources utilization by the WED-server when too many WED-workers are simultaneously in use.

%\section*{Acknowledgment}

\bibliographystyle{sbc}
\bibliography{sbc-template}



\end{document}


